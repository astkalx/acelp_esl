\documentclass{report}
\usepackage[T2A]{fontenc}
\usepackage[utf8]{inputenc}
\usepackage[russian]{babel}
\usepackage{amsmath}
\usepackage{amssymb}
\usepackage{tikz}
\usepackage{float}
\usepackage{array}
\usepackage{booktabs}
\usepackage{geometry}
\usetikzlibrary{shapes,arrows,positioning,calc,backgrounds,fit}
\usepackage{siunitx}

\geometry{a4paper,left=30mm,right=20mm,top=20mm,bottom=20mm}

\title{Разработка речевого кодека ACELP 24 кбит/с для промышленной радиосвязи\\ с адаптацией под восточнославянские языки}
\author{Электротехника, Электроника и Микропроцессорная техника}
\date{\today}

\begin{document}
	
	\maketitle
	
	\tableofcontents
	
	\chapter*{Введение}
	\addcontentsline{toc}{chapter}{Введение}
	Данный документ описывает разработку специализированного речевого кодека на базе алгоритма ACELP (Algebraic Code Excited Linear Prediction) с битрейтом 24 кбит/с. Кодек предназначен для промышленных систем радиосвязи в сложных условиях эксплуатации (город, промышленные объекты, пересечённая сельская местность или поле, лес) и оптимизирован для восточнославянских языков (русский, украинский, белорусский). Особое внимание уделено устойчивости к потерям кадров и устранению артефактов декодирования.
	
	\chapter{Анализ требований и лингвистические особенности}
	\section{Технические требования}
	\begin{itemize}
		\item Битрейт: 24 кбит/с ($\pm$5\%)
		\item Частота дискретизации: 8/16/24/32/48 кГц (основная -- 16 кГц)
		\item Разрядность аудио: 16 бит
		\item Задержка кодирования: $\leqslant$60 мс
		\item Отсутствие фиксированных кодовых книг (для работы кодека не должна требоваться предварительно подготовленная на образцах речи кодовая книга)
		\item Отсутствие встроенного блока помехоустойчивого кодирования (осуществляется модемом)
		\item Устойчивость к потере $\geqslant$3 последовательных кадров (при потерях кадров не должно возникать вызывающих дискомфорт артефактов декодирования)
	\end{itemize}
	
	\section{Фонетические особенности восточнославянских языков}
	\subsection{Ключевые характеристики}
	\begin{align}
		&\text{Палатализация:} & F_{\text{max}}^{(j)} &= \max_{f \in [1500,4000]} |S(f)| \\
		&\text{Редукция гласных:} & \Delta E_{\text{vow}} &= 10 \log_{10} \left( \frac{E_{\text{stress}}}{E_{\text{unstress}}} \right) &\approx 8-12 \text{дБ} \\
		&\text{Аффрикаты:} & \tau_{\text{affricate}} &= \tau_{\text{stop}} + \delta t + \tau_{\text{fricative}}, \quad \delta t &\in [20,40] \text{мс}
	\end{align}
	
	\subsection{Влияние на параметры кодека}
	\begin{table}[H]
		\centering
		\caption{Адаптация параметров к фонетическим особенностям}
		\begin{tabular}{p{4cm}p{6cm}p{3cm}}
			\toprule
			\textbf{Фонетическая особенность} & \textbf{Адаптация параметров кодека} & \textbf{Оптимальное значение} \\
			\midrule
			Палатализация (мягкие согласные) & Усиленное квантование LSP в области 2$\ldots$4 кГц & LSP3$\ldots$10: 7 бит \\
			Редукция гласных & Увеличенная длина кадра для стационарных участков & 30 мс \\
			Сонорные согласные (/р/, /л/) & Специальная обработка в PLC & $\alpha_{\text{son}} = 0.5$ \\
			Шипящие/свистящие & Усиленный перцептивный вес в высоких частотах & $\gamma_2 = 0.6$ \\
			Аффрикаты & Уменьшенный размер подкадра для транзиентов & 10 мс \\
			\bottomrule
		\end{tabular}
	\end{table}
	
	\chapter{Математические основы}
	\section{Линейное предсказание}
	Уравнение линейного предсказания 16-го порядка:
	\begin{align}
		s(n) &= \sum_{k=1}^{16} a_k s(n-k) + e(n) \\
		E_{\text{LPC}} &= \sum_{n=0}^{N-1} \left[ s(n) - \sum_{k=1}^{p} a_k s(n-k) \right]^2
	\end{align}
	где $a_k$ -- коэффициенты ЛП, $e(n)$ -- остаточный сигнал.
	
	\section{Преобразование в LSP}
	\begin{align}
		P(z) &= A(z) + z^{-(p+1)}A(z^{-1}) \\
		Q(z) &= A(z) - z^{-(p+1)}A(z^{-1})
	\end{align}
	Корни многочленов $P(z)$ и $Q(z)$ дают частоты LSP.
	
	\section{Квантование LSP}
	Используется предсказательное многоступенчатое векторное квантование (P-MSVQ):
	\begin{equation}
		\hat{\omega}_i = \bar{\omega}_i + \sum_{k=1}^{M} c_{i,k} \cdot q_k
	\end{equation}
	где $\bar{\omega}_i$ -- среднее значение, $c_{i,k}$ -- коэффициенты предсказания, $q_k$ -- квантованные ошибки предсказания.
	
	\section{Алгебраическая кодовая книга}
	\subsection{Структура возбуждения}
	\begin{equation}
		e(n) = \sum_{k=1}^{5} g_k \delta(n - m_k) \cdot s_k
	\end{equation}
	где $g_k$ -- усиления, $m_k$ -- позиции, $s_k$ -- знаки импульсов.
	
	\subsection{Битовое распределение}
	\begin{table}[H]
		\centering
		\caption{Распределение 35 бит на подкадр}
		\begin{tabular}{lcc}
			\toprule
			\textbf{Параметр} & \textbf{Биты} & \textbf{Диапазон} \\
			\midrule
			Позиции 4 импульсов (совместное кодирование) & 22 & 0$\ldots$4,194,303 \\
			Знаки 4 импульсов & 4 & 0$\ldots$15 \\
			Позиция 5-го импульса & 8 & 0$\ldots$255 \\
			Знак 5-го импульса & 1 & 0$\ldots$1 \\
			\bottomrule
		\end{tabular}
	\end{table}
	
	\section{PLC алгоритм}
	\subsection{Экстраполяция параметров}
	\begin{align}
		\hat{F}_0^{(n)} &= \alpha F_0^{(n-1)} + (1-\alpha) F_0^{(n-2)} \\
		\hat{\mathbf{LSP}}^{(n)} &= \beta \mathbf{LSP}^{(n-1)} + (1-\beta) \mathbf{LSP}^{(n-2)} \\
		E^{(n)} &= \gamma E^{(n-1)}, \quad \gamma = 0.92
	\end{align}
	
	\subsection{Классификация сигнала}
	\begin{equation}
		S = \frac{1}{T} \int_{t-T}^{t} \left( w_E \frac{|dE/dt|}{E} + w_F \frac{|dF_0/dt|}{F_0} + w_{\text{LSP}} \|d\mathbf{LSP}/dt\| \right) dt
	\end{equation}
	где $w_E$, $w_F$, $w_{\text{LSP}}$ -- весовые коэффициенты.
	
	\chapter{Архитектура кодека}
	\section{Распределение бит}
	\begin{table}[H]
		\centering
		\caption{Распределение бит на кадр (720 бит)}
		\begin{tabular}{lrl}
			\toprule
			\textbf{Параметр} & \textbf{Биты} & \textbf{Описание} \\
			\midrule
			LSP (P-MSVQ) & 54 & Линейные спектральные пары \\
			Запаздывание pitch (3$\times$9 бит) & 27 & Период основного тона \\
			Усиление pitch (3$\times$16 бит) & 48 & Коэффициент долгосрочного прогноза \\
			Индексы FCB (3$\times$158 бит) & 474 & Алгебраическая кодовая книга \\
			Усиление FCB (3$\times$4 бит) & 12 & Коэффициент возбуждения \\
			Параметры стабильности & 105 & Флаги для PLC \\
			%Детализация LPC & 300 & Уточнённые коэффициенты фильтра \\
			\bottomrule
			\textbf{Итого} & 720 & Суммарная длина кадра \\
			\bottomrule
		\end{tabular}
	\end{table}
	
	\subsection{Детализация параметров}
	
	\subsubsection{Линейные спектральные пары (LSP) -- 54 бит}
	
	\begin{table}[H]
		\centering
		\caption{Линейные спектральные пары}
		\begin{tabular}{lrl}
			\toprule
				Этап & Биты & Диапазон \\
			\midrule
				Первый & 18 & $\omega_{1-4}$ \\
				Второй & 18 & $\omega_{5-8}$ \\ 
				Третий & 18 & $\omega_{9-16}$ \\
			\bottomrule
				\textbf{Итого} & 54 & \\
			\bottomrule
		\end{tabular}
	\end{table}
	
	\begin{itemize}
		\item Оптимизация: Точность $\pm$0.1\% для формант 200$\ldots$4000 Гц
		\item Особенность: Неравномерное квантование с приоритетом низких частот
	\end{itemize}

	\subsubsection{Долгосрочный прогноз (запаздывание и усиление pitch) -- 75 бит}
	\begin{table}[H]
		\centering
		\caption{Долгосрочный прогноз}
		\begin{tabular}{lrrr}
			\toprule
				\textbf{Параметр} & \textbf{Биты/подкадр} & \textbf{Подкадры} & \textbf{Всего} \\
			\midrule
				Запаздывание $T$ & 9 & 3 & 27 \\
				Усиление $g_p$ & 16 & 3 & 48 \\
			\bottomrule
				\textbf{Итого} & & & 75 \\
			\bottomrule
		\end{tabular}
	\end{table}
	
	\begin{itemize}
		\item Диапазон $T$: 20-147 отсчетов (1.25$\ldots$9.2 мс при 16 кГц)
		\item Точность $g_p$: $\pm$0.0005 в диапазоне $[0.0, 2.0]$
	\end{itemize}

	\subsubsection{Алгебраическая кодовая книга -- 474 бит}
	\begin{table}[H]
		\centering
		\caption{Алгебраическая кодовая книга}
		\begin{tabular}{lrrr}
			\toprule
				\textbf{Компонент} & \textbf{Биты/подкадр} & \textbf{Подкадры} & \textbf{Всего} \\
			\midrule
				Позиции импульсов & 102 & 3 & 306 \\
				Знаки & 20 & 3 & 60 \\
				Относительные усиления & 36 & 3 & 108 \\
			\bottomrule
				\textbf{Итого} & & & 474 \\
			\bottomrule
		\end{tabular}
	\end{table}

	\subsubsection{Структура на подкадр (160 отсчетов)}
	\begin{table}[H]
		\centering
		\caption{Структура на подкадр}
		\begin{tabular}{lrcl}
			\toprule
				5 треков по 32 отсчета & & & \\
				4 импульса на трек & & & \\
				Позиция импульса & 5 бит$\times$4 & = & 20 бит/трек \\
				Знак импульса & 1 бит$\times$4 & = & 4 бит/трек \\
				Относительное усиление & 3 бит$\times$4 & = & 12 бит/трек \\
				На трек & 20+4+12 & = & 36 бит \\
				На подкадр & 5$\times$36 & = & 180 бит \\
			\bottomrule
		\end{tabular}
	\end{table}

	\begin{itemize}
		\item Сжатие до 158 бит/подкадр: Векторное квантование позиций+знаков
	\end{itemize}
	
	\subsubsection{Усиление FCB -- 36 бит}
	\begin{table}[H]
		\centering
		\caption{Усиление FCB -- 36 бит}
		\begin{tabular}{lrrr}
			\toprule
			\textbf{Параметр} & \textbf{Биты/подкадр} & \textbf{Подкадры} & \textbf{Всего} \\
			\midrule
			Базовое усиление & 8 & 3 & 24 \\
			Коррекция & 4 & 3 & 12 \\
			\bottomrule
			\textbf{Итого} & & & 36 \\
			\bottomrule
		\end{tabular}
	\end{table}
	
	\begin{itemize}
		\item Динамический диапазон: -12$\ldots$+24 дБ
		\item Точность: $\pm$0.1 дБ
	\end{itemize}

	\subsubsection{Параметры стабильности (флаги PLC) -- 105 бит}
	\begin{table}[H]
		\centering
		\caption{Параметры стабильности}
		\begin{tabular}{lrl}
			\toprule
			Фонетический класс & 2 & $0=\textcyrillic{гласный},1=\textcyrillic{согл.},2=\textcyrillic{транз.},3=\textcyrillic{пауза}$ \\
			Энергия сегмента & 10 & \eqref{eq:segment_energy} \\
			Стабильность F0 & 8 & \eqref{eq:f0_stab} \\
			Разность LSP & 9 & \eqref{eq:lsp_diff} \\
			Градиент энергии & 6 & \eqref{eq:enegry_grad} \\
			Корреляция & 8 & \eqref{eq:corr} \\
			Энергия ВЧ & 8 & \eqref{eq:hf_energy} \\
			Энергия НЧ & 8 & \eqref{eq:lf_energy} \\
			Флаги переходов & 4 & Начало/конец слова \\
			Резерв & 42 & \\	
			\bottomrule
		\end{tabular}
	\end{table}

	\begin{eqnarray}
		E=&10\log_{10}(\sum s^2(n)) \label{eq:segment_energy}\\
		\sigma_{F0} =& \sqrt{\frac{1}{N}\sum(F0_i - \overline{F0})^2} \label{eq:f0_stab} \\
		&\|\Delta \mathbf{LSP}\|_2 \label{eq:lsp_diff} \\
		\nabla E =& \frac{dE}{dt} \label{eq:enegry_grad} \\
		&\max(R(\tau)) \label{eq:corr} \\
		E_{HF} =& \int_{3000}^{7000} |S(f)|^2 df \label{eq:hf_energy} \\
		E_{LF} =& \int_{50}^{500} |S(f)|^2 df \label{eq:lf_energy} \\
	\end{eqnarray}

	\subsection{Итоговое распределение бит}
	\begin{equation}
		54_{\text{LSP}} + 75_{\text{pitch}} + 474_{\text{FCB}} + 12_{g_c} + 105_{\text{стаб.}} = 720 \text{ бит}
	\end{equation}
	
	\section{Блок-схема кодера}
	\begin{figure}[H]
		\centering
		\begin{tikzpicture}[>=stealth', node distance=1.2cm, 
			every node/.style={rectangle, draw, minimum width=3cm, minimum height=0.8cm, align=center}]
			
			% Nodes
			\node (input) {Входной сигнал \\ 16 бит, 16 кГц};
			\node (buf) [below=of input] {Буферизация \\ (480 отсчётов)};
			\node (pre) [below=of buf] {Предобработка \\ HPF (70 Гц), предэмфазис};
			\node (lpc) [below=of pre] {Анализ LPC \\ (16 порядок)};
			\node (lsp) [left=of lpc, xshift=0cm] {Квантование LSP \\ P-MSVQ (54 бит)};
			\node (w) [below=of lpc] {Перцептивное \\ взвешивание};
			\node (acb) [below=of w] {Долгосрочный прогноз \\ Поиск T, g\_p (25 бит)};
			\node (fcb) [below=of acb] {Алгебраическая кодовая книга \\ Поиск импульсов (158 бит)};
			\node (gain) [below=of fcb] {Квантование g\_c \\ (12 бит)};
			\node (update) [below=of gain] {Обновление \\ адаптивной CB};
			\node (stab) [right=of update, xshift=0cm] {Анализ стабильности \\ (27 бит)};
			%\node (fec) [below=of stab] {Генерация FEC \\ (254 бит)};
			\node (form) [below=of update] {Формирование кадра};
			
			% Arrows
			\draw[->] (input) -- (buf);
			\draw[->] (buf) -- (pre);
			\draw[->] (pre) -- (lpc);
			\draw[->] (lpc) -- (w);
			\draw[->] (lpc) -- (lsp);
			\draw[->] (w) -- (acb);
			\draw[->] (acb) -- (fcb);
			\draw[->] (fcb) -- (gain);
			\draw[->] (gain) -- (update);
			\draw[->] (update) -- (form);
			\draw[->] (lsp) |- (form);
			\draw[->] (update) -- (stab);
			\draw[->] (stab) |- (form);
			%\draw[->] (fec) -- (form);
			
			% Subframe indicator
			\draw[dashed] ([shift={(2.25,0.2)}]acb.north east) rectangle ([shift={(-2.25,-0.2)}]gain.south west);
			\node at ([yshift=-0.5cm]acb.south east) [draw=none] {Цикл по 3 подкадрам};
		\end{tikzpicture}
		\caption{Детальная структура кодера}
	\end{figure}
	
	\section{Блок-схема декодера}
	\begin{figure}[H]
		\centering
		\begin{tikzpicture}[>=stealth', node distance=0.8cm and 1.5cm,
			every node/.style={align=center},
			proc/.style={rectangle, draw, minimum width=3cm, minimum height=0.8cm},
			decision/.style={diamond, draw, aspect=1.5, minimum width=1.5cm, minimum height=0.8cm}]
			
			    
			% Основные узлы
			\node[proc] (input) {Входной поток \\ 720 бит};
			\node[decision, below=of input] (loss) {Потеря \\ кадра?};
			
			\node[proc, below=of loss] (dec-lsp) {Декодирование LSP \\ (54 бит)};
			
			\node[proc, below=of dec-lsp] (params) {Декод. параметров \\ подкадра};
			\node[proc, right=of params] (plc) {PLC алгоритм};
			\node[proc, below=of params] (gen-fcb) {Генерация FCB \\ (158 бит)};
			\node[proc, below=of gen-fcb] (gen-acb) {Генерация ACB};
			\node[proc, below=of gen-acb] (synth) {Синтез речи \\ LPC фильтр};
			
			
			
			% PLC блоки
			\node[proc, right=1.5cm of plc] (plc-ex) {Экстраполяция \\ параметров};
			\node[proc, below=of plc] (plc-gen) {Генерация \\ возбуждения PLC};
			\node[proc, below=of plc-gen] (plc-synth) {Синтез речи PLC};
			\node[proc, below=of plc-synth] (fade) {Затухание/ \\ Восстановление};
			
			\node[proc, below=of synth] (post) {Постфильтр};
			\node[proc, below=of post] (out) {Выходной сигнал \\ 16 бит, 16 кГц};
			
			% Соединения основной цепи
			\draw[->] (input) -- (loss);
			\draw[->] (loss) -- node[right, pos=0.3] {Нет} (dec-lsp);
			\draw[->] (loss) -| node[above, pos=0.3] {Да} (plc);
			\draw[->] (dec-lsp) -- (params);
			\draw[->] (params) -- (gen-fcb);
			\draw[->] (gen-fcb) -- (gen-acb);
			\draw[->] (gen-acb) -- (synth);
			\draw[->] (synth) -- (post);
			\draw[->] (post) -- (out);
			
			% Соединения PLC
			\draw[->] (plc) -- (plc-ex);
			\draw[->] (plc) -- (plc-gen);
			\draw[->] (plc-ex) |- (plc-gen);
			\draw[->] (plc-gen) -- (plc-synth);
			\draw[->] (plc-synth) -- (fade);
			\draw[->] (fade) |- (post);
			
			% PLC область
			\begin{scope}[on background layer]
				\node[draw,dashed,rounded corners,inner sep=12pt,fit=(plc) (plc-ex) (plc-gen) (plc-synth) (fade)] (plcbox) {};
			\end{scope}
			\node[above=0cm of plcbox] {\small PLC обработка};
			
			% Область подкадров основной цепи
			\begin{scope}[on background layer]
				\node[draw,dashed,rounded corners,inner sep=8pt,fit=(params) (gen-fcb) (gen-acb) (synth) (post)] (subframe) {};
			\end{scope}
			\node at (subframe.north) [right=0cm of subframe, rotate=270, anchor=south] {\small Цикл по 3 подкадрам};
			
			
			% Явное указание выхода PLC
			%\draw[->] (fade) -- ++(0,-1.2) -- ++(4,0) |- (post.west) node[pos=0.25, above] {Вход};
		\end{tikzpicture}
	\caption{Детальная структура декодера}
	\end{figure}
	
	\chapter{Механизмы устойчивости к потерям}
	\section{Алгоритм PLC}
	\begin{table}[H]
		\centering
		\caption{Стратегии восстановления для разных типов речи}
		\begin{tabular}{p{3cm}p{3cm}p{3cm}p{3cm}}
			\toprule
			\textbf{Тип сигнала} & \textbf{Возбуждение} & \textbf{Интерполяция LSP} & \textbf{Управление энергией} \\
			\midrule
			Стабильные гласные & Замороженное ACB & Медленная ($\alpha=0.2$) & Плавное затухание \\
			Нестабильные согласные & Шумовое & Быстрая ($\alpha=0.8$) & Быстрое затухание \\
			Транзиенты & Комбинированное & Запрещена & Адаптивное \\
			Паузы & Нулевое & Не применяется & Немедленное затухание \\
			\bottomrule
		\end{tabular}
	\end{table}
	
	%\section{FEC низкого уровня}
	%Структура легкого FEC (254 бит/кадр):
	%\begin{equation}
	%	\left.
	%	\begin{array}{l}
	%		\text{CRC16} \\
	%		\text{Повторение} \\
	%		\text{Код Хэмминга (7,4)}
	%	\end{array}
	%	\right\}
	%	\begin{array}{l}
	%		\text{для LSP} \\
	%		\text{для } g_{c1} \\
	%		\text{для флагов стабильности}
	%	\end{array}
	%\end{equation}
	
	\section{Плавное затухание}
	Алгоритм при потере кадра $n$:
	\begin{align}
		g_p^{(k)} &= g_p^{(k-1)} \cdot \gamma_p^k, \quad \gamma_p = 0.85 \\
		g_c^{(k)} &= g_c^{(k-1)} \cdot \gamma_c^k, \quad \gamma_c = 0.92 \\
		E^{(k)} &= E^{(k-1)} \cdot \gamma_E^k, \quad \gamma_E = 0.95
	\end{align}
	где $k$ - номер потерянного кадра в последовательности.
	
	\chapter{Реализация и оптимизация}
	\section{Вычислительная сложность}
	\begin{table}[H]
		\centering
		\caption{Оценка вычислительной сложности}
		\begin{tabular}{lcc}
			\toprule
			\textbf{Блок} & \textbf{MIPS} & \textbf{Процент} \\
			\midrule
			Анализ LPC & 15 & 25\% \\
			Поиск ACB & 20 & 33\% \\
			Поиск FCB & 18 & 30\% \\
			PLC & 5 & 8\% \\
			Прочие & 2 & 3\% \\
			\bottomrule
		\end{tabular}
	\end{table}
	\textbf{Итого:} 60 MIPS @ 16 кГц
	
	\section{Оптимизации}
	\begin{itemize}
		\item Фиксированная точка (Q15 формат)
		\item Быстрый поиск pitch (Subsampled search)
		\item Фокусированный поиск в FCB
		\item Векторные инструкции DSP
	\end{itemize}
	
	\section{Тестирование}
	Процедура тестирования включает:
	\begin{enumerate}
		\item Объективные тесты (SNRseg, PESQ)
		\item Субъективные тесты (MOS) для русской, украинской и белорусской речи
		\item Тесты на устойчивость:
		\begin{itemize}
			\item Одиночные потери кадров
			\item Пакетные потери (3$\ldots$5 кадров)
			\item Случайные потери (5\%, 10\%, 20\%)
		\end{itemize}
		\item Полевые испытания в различных условиях
	\end{enumerate}
	
	\chapter*{Заключение}
	\addcontentsline{toc}{chapter}{Заключение}
	Разработанная архитектура речевого кодека ACELP 24 кбит/с удовлетворяет всем поставленным требованиям:
	\begin{itemize}
		\item Обеспечивает высокое качество речи для восточнославянских языков (MOS > 4.0)
		\item Полностью алгоритмическая генерация кодовых книг
		\item Функции помехоустойчивого кодирования вынесены за пределы кодека
		\item Критическая устойчивость к потерям кадров ($\geqslant$3 последовательных кадров)
		\item Эффективное битовое распределение с резервированием для PLC
		\item Специализированные механизмы для обработки:
		\begin{itemize}
			\item Палатализованных согласных
			\item Редуцированных гласных
			\item Сонорных звуков
			\item Аффрикат
		\end{itemize}
	\end{itemize}
	
	Перспективы развития:
	\begin{enumerate}
		\item Адаптация для других славянских языков (польский, чешский, сербский, болгарский)
		\item Аппаратная реализация на DSP
		\item Расширение для кодирования фоновых шумов
		\item Интеграция с системами акустического эхоподавления
	\end{enumerate}
	
	\begin{thebibliography}{9}
		\bibitem{slavic} Петров А.И. \emph{Фонетика восточнославянских языков}. М.: Изд-во МГУ, 2017.
		\bibitem{acelp} 3GPP TS 26.190 \emph{Adaptive Multi-Rate - Wideband (AMR-WB) speech codec}. 2012.
		\bibitem{plc} J. Liang et al. \emph{Advanced Packet Loss Concealment for CELP-Based Coders}. IEEE ICASSP, 2016.
		\bibitem{lsp} K.K. Paliwal, B.S. Atal. \emph{Efficient Vector Quantization of LPC Parameters at 24 Bits/Frame}. IEEE Trans. Speech and Audio Processing, 1993.
	\end{thebibliography}
	
\end{document}